\documentclass[12pt]{article}
\usepackage[a4paper, heightrounded,
	bindingoffset=6mm, left=10mm, right=40mm, marginparsep=5mm, marginparwidth=32.5mm,
	headheight=15pt, top=25mm, bottom=20mm]{geometry}
\usepackage{amsmath}
\usepackage{amsthm}
\usepackage{xcolor}
\usepackage{setspace}

\usepackage{tikz}
\newcommand{\contradiction}{\tikz[baseline, x=0.22em, y=0.22em, line width=0.032em]\draw (0,2.83)--(2.83,0) (0.71,3.54)--(3.54,0.71) (0,0.71)--(2.83,3.54) (0.71,0)--(3.54,2.83);}

\let\phi\varphi
\let\from\gets
\def\ofA{\langle A \rangle}
\renewcommand{\a}[1]{\alpha_#1}

\newcommand\alert\textbf
\newcommand\todo[1]{{\color{red}TODO: #1}}

\usepackage{marginnote}
\usepackage{ragged2e}
\renewcommand*{\marginfont}{\normalfont\footnotesize}
\renewcommand*{\raggedleftmarginnote}{\RaggedRight}
\renewcommand*{\raggedrightmarginnote}{\RaggedRight}

\begin{document}

\paragraph{Claim} Let $\phi \in G_{n,r}$ be in quasinormal form with respect to $X$. If $y\in X$ belongs to an $X$-component of type (B), then the statement $$y \phi^{i}\in X\ofA \implies y\phi^{i} \in X$$ holds when
\begin{itemize}
	\item $i < 0$, if the $X$-component of $y$ is right semi-infinite;
	\item $i > 0$, if the $X$-component of $y$ is left semi-infinite.
\end{itemize}

\begin{proof}[Proof for RSI component] Let $i < 0$ be chosen so that $y\phi^i \in X\ofA$. By way of contradiction, suppose $y \phi^{i} \not\in X$. First, we claim that $y \phi^{i}$ cannot be above $X$. If this were so, we would have expanded a candidate quasinormal basis $X'$ at $y \phi^{i}$ in the search for the true quasinormal basis $X$. But we would not do this: we only expand when we find words in incomplete $X'$-components, and $y \phi^{i}$ is in a semi-infinite $X$-component. \marginnote{DMR: Not completely convinced of the last assertion in the first paragraph.}

Thus $y \phi^{i} = w\Delta$ for some $w \in X$, with $\Delta \neq 1$ by our supposition. If $w$ had type (A), then so would $y\phi^i$ and hence so would $y$. The same is true replacing (A) with (C). Finally, $w \in X$, so $w$ cannot belong to an incomplete X-component. The only choice remaining is that $w$ has type (B).

Let $y$, $w$ have characteristic powers $a$, $b$ respectively. We must have $\mathrm{sign}(b) = \mathrm{sign}(a)$, because otherwise $y$ would belong to a doubly infinite (type (C)) $X$-component. Thus $w$ belongs to a right semi-infinite $X$-component, meaning that $w\phi^j \in X\ofA$ for all $j > 0$.

Now choose $j = -i > 0$ to obtain $w\phi^{-i} = z\Lambda \in X\ofA$, for some $z \in X$. Observe that $$z\Lambda\Delta = w\phi^{-i}\Delta = w\Delta\phi^{-i} = y\phi^i\phi^{-i} = y \in X,$$
from which we conclude that $z = y$ and $\Lambda = \Delta = 1$. But we observed earlier that $\Delta \neq 1$; this is a contradiction. \contradiction
\end{proof}

\end{document}