\documentclass[12pt]{article}
\usepackage[a4paper, heightrounded,
	bindingoffset=6mm, left=10mm, right=40mm, marginparsep=5mm, marginparwidth=32.5mm,
	headheight=15pt, top=25mm, bottom=20mm]{geometry}
\usepackage{amsmath}
\usepackage{amsthm}
\usepackage{xcolor}
\usepackage{setspace}

\let\phi\varphi
\let\from\gets
\def\ofA{\langle A \rangle}
\renewcommand{\a}[1]{\alpha_#1}

\newcommand\alert\textbf
\newcommand\todo[1]{{\color{red}TODO: #1}}

\usepackage{marginnote}
\usepackage{ragged2e}
\renewcommand*{\marginfont}{\normalfont\footnotesize}
\renewcommand*{\raggedleftmarginnote}{\RaggedRight}
\renewcommand*{\raggedrightmarginnote}{\RaggedRight}

\begin{document}

\paragraph{Claim} Let $\phi \in G_{n,r}$ be in quasinormal form with respect to $X$. If $y\in X$ belongs to an $X$-component of type (B), then the statement $$y \phi^{i}\in X\ofA \implies y\phi^{i} \in X$$ holds when
\begin{enumerate}
	\item $i < 0$, if the $X$-component of $y$ is right semi-infinite;
	\item $i > 0$, if the $X$-component of $y$ is left semi-infinite.
\end{enumerate}
\begin{proof}[Proof for RSI component] Suppose $y \phi^{i} \in X\ofA \setminus X$ with $i < 0$. We know that $y \phi^{i}$ cannot be above $X$: if so we would have expanded a candidate quasinormal basis at $y \phi^{i}$. This is not something we would $y \phi^{i}$ has type (B).

Thus $y \phi^{i} = w\Delta$ for some $w \in X$. If $w$ had type (A), then so would $y\phi^i$ and hence so would $y$. The same is true replacing (A) with (C). Finally, $w \in X$, so $w$ cannot belong to an incomplete X-component. Thus $w$ is of type (B).

Suppose $y$ has characteristic power $a$ and suppose $w$ has characteristic power $b$. We cannot have $\mathrm{sign}(b) \neq \mathrm{sign}(a)$, because then $y$ would belong to a doubly infinite orbit (neccesarily of type (C)). Thus $w$ belongs to a right semi-infinite $X$-component. 

\end{proof}

\end{document}